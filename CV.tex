
% LaTeX Template
% Version 1.1 (9/12/12)
%
% This template has been downloaded from:
% http://www.LaTeXTemplates.com
%
% Original author:
% Rensselaer Polytechnic Institute (http://www.rpi.edu/dept/arc/training/latex/resumes/)
%
% Important note:
% This template requires the res.cls file to be in the same directory as the
% .tex file. The res.cls file provides the resume style used for structuring the
% document.
%
%%%%%%%%%%%%%%%%%%%%%%%%%%%%%%%%%%%%%%%%%

%----------------------------------------------------------------------------------------
%	PACKAGES AND OTHER DOCUMENT CONFIGURATIONS
%----------------------------------------------------------------------------------------

\documentclass[margin, 10pt]{res} % Use the res.cls style, the font size can be changed to 11pt or 12pt here
\topmargin=-0.50in
\oddsidemargin -0.25in
\evensidemargin -.25in
\textwidth=8.0in
\itemsep=0in
\parsep=0in
\footskip= -0.0in
\textheight= 11.0in
\usepackage{helvet} % Default font is the helvetica postscript font
%\usepackage{newcent} % To change the default font to the new century schoolbook postscript font uncomment this line and comment the one above
\usepackage{amsmath}
\usepackage{fancyhdr}
\usepackage{hyperref}
\usepackage{graphicx}
\pagestyle{fancy}
\fancyhf{}
\lfoot{\mbox{\kern-3.5cm$\text{*}$ as on \today}}

\cfoot{\hspace{7.5cm}  Page:\thepage \hspace{0.1cm}of 2}


\renewcommand{\headrulewidth}{0pt}

\setlength{\textwidth}{5.1in} % Text width of the document
% Set the headings' appearance in the ``fancy'' pagestyle
 
\begin{document}

%----------------------------------------------------------------------------------------
%	NAME AND ADDRESS SECTION
%----------------------------------------------------------------------------------------

\moveleft.5\hoffset\centerline{\LARGE\bf Vibhor Aggarwal} % Your name at the top
 \moveleft.5\hoffset\centerline{Master's Student, Automotive Engineering at RWTH Aachen}
 
% { Research fellow, Italian Institute of Technology, Genova, Italy }
\moveleft\hoffset\vbox{\hrule width\resumewidth height 1pt}\smallskip % Horizontal line after name; adjust line thickness by changing the '1pt'

\moveleft.5\hoffset\leftline {vibhorag24@gmail.com  \hspace{9cm}Bayernallee 7}  % Your address
\moveleft.5\hoffset\leftline {(+49)176-598-96222\hspace{9cm} Aachen,Germany}
\moveleft.5\hoffset\leftline {Webpage:\href{https://vibhoraggarwal.github.io/}{vibhoraggarwal.github.io/}}
%
%\moveleft.5\hoffset\centerline{vibhorag24@gmail.com   \hspace{0.75cm}      (+39) 328-658-2811}
%\moveleft.5\hoffset\centerline{vibhorag24@gmail.com}

%----------------------------------------------------------------------------------------

\begin{resume}
%----------------------------------------------------------------------------------------
%	OBJECTIVE SECTION
%----------------------------------------------------------------------------------------

\section{ACHIEVEMENT AND AWARDS} 


\textbf{Best Under-Graduate project} in Mechanical Engineering department of IIT Kanpur in 2017 \\
\textbf{Ranjan Kumar Memorial Award} for the \textbf{best socially relevant project} at IIT Kanpur in 2017\\
Ranked \textbf{3rd} in state and 914 nationally \textbf{among 1.4 million students} in JEE 2013\\
%Ranked 103 in National level Science Talent  Search Examination by Unified Council\\
\textbf{Gold Medal} in National Mathematics Olympiad conducted by AISMTA,2013\\


%----------------------------------------------------------------------------------------
%	EDUCATION SECTION
%----------------------------------------------------------------------------------------

\section{EDUCATION}

{\sl M.Sc, Automotive Engineering }%\hfill CGPA -/-
\hfill Sep, 2018 - Present \\ 
RWTH Aachen, Germany

{\sl B.Tech, Major: Mechanical Engineering }%\hfill CGPA 6.6/10
  \hfill July 2013 - June 2017 \\ 
\phantom{x}\hspace{6.5ex} {\sl   Minor: Applied Mathematics}\\IIT Kanpur, India

 

 {\sl Intermediate, Central Board of Secondary Education} \hfill %\hfill 95.6\% 
 \hfill May 2013 \\
Army Public School, Dehradun, India


%{\sl Matric, Central Board of Secondary Education} %\hfill\hfill\hfill\hfill\hfill \hfill\hfill CGPA: 10/10 
%\hfill May 2011 \\
%St. Mary's Convent School, Dehradun, India
 

%----------------------------------------------------------------------------------------
%	PROFESSIONAL EXPERIENCE SECTION
%----------------------------------------------------------------------------------------
\section{EXPERIENCE}

{\sl Research Fellow, }Dynamic Interaction Control \hfill Nov 2017 - Aug 2018 \\
Supervisor: Dr. Daniele Pucci, Italian Institute of Technology, Genova, Italy\\

\begin{itemize}
	\item  Defining and identifying the \textbf{transfer function} between the voltage applied to the motors and the torque of each joint of the humanoid robot iCub
	\item  Implement the low level torque control framework on the joints of iCub using the identified transfer function

\end{itemize} 

{\sl Graduate Engineer Trainee} \hfill July 2017 - Oct 2017 \\
Hero Motocorp Ltd, Haridwar, India\\

\begin{itemize}
	\item Managed \textbf{Total productive Maintenance} for machinery
	equipment and quality related activities and completing operations pertaining to maintenance repair
	involving resource planning and in-process inspection
	\item Produced machined parts by programming, setting up, and operating a computer numerical control (CNC) machine;
	maintaining quality and safety standards

	
\end{itemize} 
{\sl Intern, Mechanical Design Engineer} \hfill May 2016 - July 2016 \\
Grey Orange Robotics Pte. Ltd, Gurugram, India\\

\begin{itemize}
	\item Worked on the Suspension system of a robot, which is a bi- directionally scalable material handling system for goods to man
	\item Optimized the assembly through introduction of trailing link in the Suspension system, and reduced the number of parts
	\item Studied the designing of structural features and material choice for the casting of Aluminum parts

	
\end{itemize} 
\clearpage

%----------------------------------------------------------------------------------------
%	Academic projects section
%---------------------------------------------------------------------------------------- 

\section{ACADEMIC PROJECTS} 
{\sl{Robotic Exoskeleton Arm}  } \hfill Aug 2016 - April 2017 \\
Supervisor: Dr. Sumit Basu, IIT Kanpur, India\\
\begin{itemize}

\item \textbf{Exoskeleton arm} that increases mobility and is easily controlled by voice using an Android app, Bluetooth module and arduino.
\item Actuated using \textbf{Pneumatic Air Muscles}(PAM) made of Latex material,used as a woven shell and Polyethylene Terephthalate, used for loose weave working on the principle of proportional pressure  pneumatics
\item Simulated the non-linear model on Ansys, and tested it experimentally 
%\item Structure of the project is made completely of Glass Fibre Reinforced Plastic resulting 
\item Helps people affected from \textbf{Cerebral Palsy} and old age arm weakness
\end{itemize}


{\sl Multiple Crop Planting Machine} \hfill Jan 2015 - April 2015 \\
Guide: Dr. V.K Jain, IIT Kanpur, India\\
\begin{itemize} % Reduce space between items
\item Worked in a team of 7 members to design and fabricate a working model of crop planting machine
\item Simulated the machine on Solidworks and Ansys, for functionality 

\end{itemize}


%----------------------------------------------------------------------------------------
%	Co-Scholastic projects section
%---------------------------------------------------------------------------------------- 

\section{CO- SCHOLASTIC PROJECT} 
{\sl{Design and fabrication of two off-road vehicles}  } \hfill Dec 2013 - Jan 2016 \\
Supervisor: Dr. Avinash Kumar Agarwal, IIT Kanpur\\
\begin{itemize}
\item Calculated and optimized the Suspension parameters for the vehicle on “Lotus Suspension Simulation”
%\item Designed the front hub of the vehicle on Solidworks and did its FEA and weight optimization on Ansys
\item Developed a Mathematical model on MATLAB for the vehicle's Suspension system to calculate forces 
\item The project was awarded 4th position for its design among 44 national teams 
%\item Team won the "Best Technical Ready team" in 2016 for "B16" at Baja Student India conducted by Delta Shootout Inc
%\item The vehicle "B16" was awarded 5th for its design and 4th in Acceleration among 43 national teams
\end{itemize}





%----------------------------------------------------------------------------------------
%	Technology SKILLS SECTION
%----------------------------------------------------------------------------------------

\section{TECHNICAL \\ SKILLS} 

MATLAB,Solidworks,Ansys,Autodesk Inventor, Lotus Suspension Analysis, Abaqus FEA, C language\\

\section{LANGUAGES} 

German(B1), Italian(A1), English(Bilingual), Hindi(Native)\\
\section{RELEVANT \\ COURSES}
Design for Manufacturing and Assembly \hfill Fracture and Fatigue            \\
Solar Energy Technology \hfill Design of Machine Elements\\
Mechanics of Solids \hfill Additive Manufacturing\\
Theory of Mechanisms and machine \hfill Finite Element Methods\\
Organizational and administrative psychology\hfill Mathematical Modelling\\
Mathematical Methods\hfill Vibration and Control



 






\section{ POSITIONS \\HELD}

{\sl Team Captain} \hfill April, 2015 - Jan, 2016\\
BAJA SAE, Motorsports team of IIT Kanpur\\ 
\begin{itemize} \itemsep -2pt

\item Spearheaded a team of 25 members in design and fabrication of an All-terrain vehicle for Baja Student India 2016
\item Laid the groundwork for IITK Motorsports to acquire the recognition of an institute team from  2016
\item Contacted firms like Bosch, Fox suspensions, Wilwoods, Dassault Systems etc. thereby  raising sponsorship

\end{itemize}

\clearpage

\end{resume}

\end{document}




